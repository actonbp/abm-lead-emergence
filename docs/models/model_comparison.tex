\documentclass[12pt]{article}

\usepackage{amsmath}
\usepackage{graphicx}
\usepackage{hyperref}
\usepackage{booktabs}
\usepackage{float}
\usepackage{xcolor}
\usepackage{colortbl}
\usepackage{multirow}
\usepackage[table]{xcolor}
\usepackage{array}
\usepackage{longtable}
\usepackage{makecell}

% Define colors
\definecolor{headerblue}{RGB}{51, 102, 153}
\definecolor{lightgray}{RGB}{242, 242, 242}
\definecolor{darkblue}{RGB}{28, 63, 99}

% Custom table styles
\newcolumntype{L}[1]{>{\raggedright\arraybackslash}p{#1}}
\newcolumntype{C}[1]{>{\centering\arraybackslash}p{#1}}

\title{Leadership Emergence Model Comparison}
\author{Model Documentation}
\date{\today}

\begin{document}

\maketitle
\tableofcontents
\newpage

\section{Model Overview}
This document provides a detailed comparison of leadership emergence models, starting with the base model and showing how variations build upon it. Each model component is described with its theoretical basis, implementation details, and validation metrics.

\section{Model Components}

\subsection{Agent Elements}
\begin{table}[H]
\rowcolors{2}{lightgray}{white}
\begin{tabular}{L{3cm}|L{5cm}|L{5cm}|L{3cm}}
\rowcolor{headerblue}
\textcolor{white}{\textbf{Element}} & \textcolor{white}{\textbf{Base Model (A)}} & \textcolor{white}{\textbf{Model B}} & \textcolor{white}{\textbf{Theoretical Basis}} \\
\hline
Leadership Characteristics & Leadership Characteristics agent i & ... & Social-cognitive \\
ILT & ILT for agent i & ILT represents empirical distribution & Leadership categorization \\
Leader Identity & Leader identity for agent i & ... & Identity theory \\
Follower Identity & Follower identity for agent i & ... & Role theory \\
\end{tabular}
\caption{Agent Elements Comparison}
\end{table}

\subsection{Element Assumptions}
\begin{table}[H]
\rowcolors{2}{lightgray}{white}
\begin{tabular}{L{3cm}|L{5cm}|L{5cm}|L{3cm}}
\rowcolor{headerblue}
\textcolor{white}{\textbf{Rule}} & \textcolor{white}{\textbf{Base Model (A)}} & \textcolor{white}{\textbf{Model B}} & \textcolor{white}{\textbf{Validation}} \\
\hline
Rule 1 & Leader/Follower identity is uniform distribution & ... & Distribution tests \\
Rule 2 & ILT represents uniform dist. cutoff & ILT represents distribution based on empirical literature & Empirical fit \\
Rule 3 & Leader characteristic represents uniform dist. cutoff & ... & Distribution tests \\
\end{tabular}
\caption{Element Assumptions Comparison}
\end{table}

\subsection{Interactional Rules}
\begin{table}[H]
\rowcolors{2}{lightgray}{white}
\begin{tabular}{L{3cm}|L{5cm}|L{5cm}|L{3cm}}
\rowcolor{headerblue}
\textcolor{white}{\textbf{Rule}} & \textcolor{white}{\textbf{Base Model (A)}} & \textcolor{white}{\textbf{Model B}} & \textcolor{white}{\textbf{Mechanism}} \\
\hline
Rule 1 & Only two interactants at a time & ... & Dyadic interaction \\
Rule 2 & Agents first claim, and then grant & ... & Sequential process \\
Rule 3 & Grant: Agents compare other agent's leader characteristic to their ILT & ... & Recognition process \\
Rule 4 & Claim: Agent's claim based on probabilistic cutoff from Leader identity & ... & Identity expression \\
\end{tabular}
\caption{Interactional Rules Comparison}
\end{table}

\subsection{Environmental Context}
\begin{table}[H]
\rowcolors{2}{lightgray}{white}
\begin{tabular}{L{3cm}|L{5cm}|L{5cm}|L{3cm}}
\rowcolor{headerblue}
\textcolor{white}{\textbf{Assumption}} & \textcolor{white}{\textbf{Base Model (A)}} & \textcolor{white}{\textbf{Model B}} & \textcolor{white}{\textbf{Impact}} \\
\hline
Environmental 1 & There are four agents & ... & Group dynamics \\
Environmental 2 & There is outside task or objective beyond the interactions & ... & Task context \\
\end{tabular}
\caption{Environmental Context Comparison}
\end{table}

\section{Parameter Details}

\subsection{Distribution Parameters}
\begin{table}[H]
\rowcolors{2}{lightgray}{white}
\begin{tabular}{L{3cm}|L{2cm}|L{3cm}|L{4cm}|L{3cm}}
\rowcolor{headerblue}
\textcolor{white}{\textbf{Parameter}} & \textcolor{white}{\textbf{Range}} & \textcolor{white}{\textbf{Default}} & \textcolor{white}{\textbf{Description}} & \textcolor{white}{\textbf{Sensitivity}} \\
\hline
Identity Distribution & [0, 1] & Uniform & Initial identity values & High \\
ILT Distribution & [0, 1] & Model specific & Leadership prototype & High \\
Characteristic Distribution & [0, 1] & Uniform & Leadership traits & Medium \\
\end{tabular}
\caption{Distribution Parameters}
\end{table}

\subsection{Interaction Parameters}
\begin{table}[H]
\rowcolors{2}{lightgray}{white}
\begin{tabular}{L{3cm}|L{2cm}|L{3cm}|L{4cm}|L{3cm}}
\rowcolor{headerblue}
\textcolor{white}{\textbf{Parameter}} & \textcolor{white}{\textbf{Range}} & \textcolor{white}{\textbf{Default}} & \textcolor{white}{\textbf{Description}} & \textcolor{white}{\textbf{Sensitivity}} \\
\hline
Claim Threshold & [0, 1] & 0.5 & Minimum identity for claim & High \\
Grant Threshold & [0, 1] & 0.5 & Minimum match for grant & High \\
Update Rate & [0, 1] & 0.1 & Identity update speed & Medium \\
\end{tabular}
\caption{Interaction Parameters}
\end{table}

\section{Validation Metrics}

\subsection{Pattern Metrics}
\begin{itemize}
\item \textbf{Emergence Speed}: Time steps until stable leadership structure
\item \textbf{Structure Stability}: Variance in leadership roles over time
\item \textbf{Role Distribution}: Distribution of leadership claims/grants
\item \textbf{Interaction Patterns}: Network analysis of claim/grant patterns
\end{itemize}

\subsection{Theoretical Predictions}
\begin{itemize}
\item \textbf{Base Model}:
  \begin{itemize}
  \item Gradual emergence through repeated interactions
  \item Stable leadership structure over time
  \item Role differentiation based on initial conditions
  \end{itemize}
\item \textbf{Model B}:
  \begin{itemize}
  \item Faster emergence due to empirical ILT distribution
  \item More consistent with observed leadership patterns
  \item Better alignment with theoretical predictions
  \end{itemize}
\end{itemize}

\section{Implementation Notes}

\subsection{Key Classes}
\begin{itemize}
\item \textbf{Agent}: Implements individual characteristics and behaviors
\item \textbf{Model}: Manages simulation environment and interactions
\item \textbf{Metrics}: Calculates validation metrics and patterns
\end{itemize}

\subsection{Simulation Flow}
1. Initialize agents with distributions
2. Select interaction pairs
3. Process claims and grants
4. Update identities
5. Record metrics
6. Repeat until convergence

\end{document} 