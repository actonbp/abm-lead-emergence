\documentclass[12pt]{article}

\usepackage{amsmath}
\usepackage{graphicx}
\usepackage{hyperref}
\usepackage{booktabs}
\usepackage{float}
\usepackage{xcolor}
\usepackage{tikz}
\usepackage[edges]{forest}
\usepackage{listings}

% Define colors
\definecolor{lightblue}{RGB}{235, 245, 255}
\definecolor{darkblue}{RGB}{0, 60, 120}
\definecolor{lightgray}{RGB}{240, 240, 240}

% Code listing style
\lstset{
    backgroundcolor=\color{lightgray},
    basicstyle=\ttfamily\small,
    breaklines=true,
    captionpos=b,
    commentstyle=\color{green!60!black},
    keywordstyle=\color{blue},
    showstringspaces=false,
    stringstyle=\color{red},
    numbers=left,
    numberstyle=\tiny,
    numbersep=5pt
}

\title{Leadership Emergence Model Pipeline\\
\large Implementation and Analysis Framework}
\author{Model Documentation}
\date{\today}

\begin{document}

\maketitle
\tableofcontents
\newpage

\section{Overview}
This document outlines our systematic approach to analyzing leadership emergence through a combination of agent-based modeling (ABM) and machine learning (ML) techniques. The framework enables comprehensive exploration, validation, and comparison of different theoretical perspectives on leadership emergence.

\section{Theoretical Background}

\subsection{Leadership Emergence}
Leadership emergence is a dynamic social process where individuals within a group come to be recognized as leaders through repeated interactions and mutual influence processes. Three key theoretical perspectives explain this phenomenon:

\begin{itemize}
    \item \textbf{Social-Interactionist Perspective (SIP)}: Emphasizes the role of repeated claims and grants of leadership, where leadership emerges through negotiated interactions between group members.
    
    \item \textbf{Social-Cognitive Perspective (SCP)}: Focuses on cognitive schemas and prototypes, where leadership recognition depends on the match between an individual's characteristics and others' implicit leadership theories.
    
    \item \textbf{Social Identity Theory (SIT)}: Highlights the importance of group identity and prototypicality, where leadership emerges based on how well individuals represent the group's collective identity.
\end{itemize}

\subsection{Key Emergence Outcomes}
We focus on three critical outcomes to validate theoretical predictions:

\begin{enumerate}
    \item \textbf{Emergence Speed}
    \begin{itemize}
        \item How quickly stable leadership structures form
        \item Measured by convergence time of leadership recognition patterns
        \item Theoretical predictions vary:
            \begin{itemize}
                \item SIP: Gradual emergence through repeated interactions
                \item SCP: Rapid emergence when clear prototype matches exist
                \item SIT: Moderate speed, dependent on group identity formation
            \end{itemize}
    \end{itemize}

    \item \textbf{Structure Stability}
    \begin{itemize}
        \item How stable the emerged leadership structure remains
        \item Measured by variance in leadership recognition over time
        \item Theoretical predictions:
            \begin{itemize}
                \item SIP: High stability once established through interactions
                \item SCP: Moderate stability, sensitive to context changes
                \item SIT: Very high stability when aligned with group identity
            \end{itemize}
    \end{itemize}

    \item \textbf{Distribution Patterns}
    \begin{itemize}
        \item How leadership recognition is distributed across the group
        \item Measured by network centralization and role differentiation
        \item Theoretical predictions:
            \begin{itemize}
                \item SIP: Emergent hierarchy based on interaction history
                \item SCP: Concentration around prototype-matching individuals
                \item SIT: Distribution reflecting group prototype alignment
            \end{itemize}
    \end{itemize}
\end{enumerate}

\section{Agent-Based Modeling Approach}

\subsection{Implementation Framework}
The models are implemented in Python, utilizing object-oriented programming for clear component separation and extensibility. Key libraries include:

\begin{itemize}
    \item \texttt{NumPy} for numerical computations
    \item \texttt{NetworkX} for social network analysis
    \item \texttt{scikit-learn} for pattern analysis
    \item \texttt{Pandas} for data management
\end{itemize}

\subsection{Base Model Structure}
The base model implements the core leadership emergence mechanisms in Python. Here's the high-level pseudocode:

\begin{lstlisting}[language=Python, caption=Base Model Pseudocode]
class Agent:
    def __init__(self):
        # Individual characteristics
        self.leadership_traits = initialize_traits()
        self.ilt = initialize_ilt()
        self.leader_identity = initialize_identity()
        self.follower_identity = initialize_identity()
        
    def decide_claim(self):
        # Probabilistic decision based on leader identity
        return probability > self.claim_threshold
        
    def evaluate_grant(self, other_agent):
        # Compare other's traits to ILT
        match = compare_traits_to_ilt(
            other_agent.leadership_traits)
        return match > self.grant_threshold
        
    def update_identities(self, interaction_result):
        # Update based on interaction outcome
        if interaction_result.successful:
            adjust_identities_positively()
        else:
            adjust_identities_negatively()

class LeadershipEmergenceModel:
    def __init__(self, n_agents, parameters):
        self.agents = [Agent() for _ in range(n_agents)]
        self.parameters = parameters
        self.interaction_history = []
        
    def step(self):
        # Single simulation step
        pair = select_interaction_pair()
        claimer, evaluator = pair
        
        if claimer.decide_claim():
            grant = evaluator.evaluate_grant(claimer)
            record_interaction(claimer, evaluator, grant)
            update_agents(claimer, evaluator, grant)
            
    def run_simulation(self, n_steps):
        for _ in range(n_steps):
            self.step()
            if self.check_convergence():
                break
                
        return analyze_emergence_patterns()
\end{lstlisting}

\section{Model Architecture}

\subsection{Base Model Foundation}
The base model implements core leadership emergence mechanisms:
\begin{itemize}
    \item Agent characteristics (leadership traits, ILT)
    \item Identity components (leader/follower identities)
    \item Interaction rules (claims and grants)
    \item Environmental context
\end{itemize}

\subsection{Theoretical Extensions}
Building on the base model:
\begin{enumerate}
    \item \textbf{Social-Interactionist (SIP)}
    \begin{itemize}
        \item Claims/grants process
        \item Identity negotiation
        \item Interaction patterns
    \end{itemize}
    
    \item \textbf{Social-Cognitive (SCP)}
    \begin{itemize}
        \item Schema activation
        \item Prototype matching
        \item Information processing
    \end{itemize}
    
    \item \textbf{Social Identity (SI)}
    \begin{itemize}
        \item Group prototypicality
        \item Collective identity
        \item Group dynamics
    \end{itemize}
\end{enumerate}

\section{Analysis Pipeline}

\subsection{Parameter Space Exploration}
\begin{lstlisting}[language=Python, caption=Parameter Space Definition]
parameter_space = {
    'group_size': (4, 50),
    'interaction_rate': (0.1, 1.0),
    'identity_threshold': (0.3, 0.7),
    'update_rate': (0.1, 0.5),
    'schema_weight': (0.0, 1.0),
    'prototype_similarity': (0.5, 0.9)
}
\end{lstlisting}

\subsection{ML-Driven Analysis}
\begin{enumerate}
    \item \textbf{Latin Hypercube Sampling}
    \begin{itemize}
        \item Efficient parameter space coverage
        \item Balanced sampling across dimensions
        \item Sensitivity analysis preparation
    \end{itemize}

    \item \textbf{Bayesian Optimization}
    \begin{itemize}
        \item Identify optimal parameter regions
        \item Guide exploration based on objectives
        \item Balance exploration/exploitation
    \end{itemize}

    \item \textbf{Pattern Recognition}
    \begin{itemize}
        \item Cluster analysis of emergence patterns
        \item Feature importance analysis
        \item Theory-aligned pattern detection
    \end{itemize}
\end{enumerate}

\section{Validation Framework}

\subsection{Theoretical Validation}
\begin{enumerate}
    \item \textbf{Pattern Matching}
    \begin{itemize}
        \item Compare to theoretical predictions
        \item Identify emergence mechanisms
        \item Validate causal pathways
    \end{itemize}

    \item \textbf{Cross-Theory Comparison}
    \begin{itemize}
        \item Nested model analysis
        \item Component contribution assessment
        \item Theory integration insights
    \end{itemize}
\end{enumerate}

\subsection{Empirical Validation}
\begin{enumerate}
    \item \textbf{Pattern Validation}
    \begin{itemize}
        \item Compare to empirical studies
        \item Assess emergence timelines
        \item Validate role distributions
    \end{itemize}

    \item \textbf{Parameter Calibration}
    \begin{itemize}
        \item Fit to empirical data
        \item Cross-validation
        \item Robustness testing
    \end{itemize}
\end{enumerate}

\section{Implementation}

\subsection{Nested Model Framework}
\begin{lstlisting}[language=Python, caption=Nested Model Implementation]
class NestedLeadershipModel:
    def __init__(self, components=None):
        self.base = BaseModel()
        self.components = components or []
        
    def add_component(self, component):
        self.components.append(component)
        
    def run_simulation(self):
        results = []
        for config in self.generate_configs():
            result = self.simulate(config)
            results.append(result)
        return results
\end{lstlisting}

\subsection{ML Analysis Pipeline}
\begin{lstlisting}[language=Python, caption=ML Analysis Implementation]
class MLPipeline:
    def __init__(self):
        self.sampler = LatinHypercubeSampler()
        self.optimizer = BayesianOptimizer()
        self.analyzer = PatternAnalyzer()
        
    def run_analysis(self, model):
        samples = self.sampler.sample(
            parameter_space)
        results = model.run_batch(samples)
        patterns = self.analyzer.find_patterns(
            results)
        return self.optimizer.optimize(
            patterns)
\end{lstlisting}

\section{Expected Outcomes}

\subsection{Theoretical Insights}
\begin{itemize}
    \item Mechanism importance ranking
    \item Theory integration opportunities
    \item Context dependency patterns
    \item Novel theoretical predictions
\end{itemize}

\subsection{Methodological Contributions}
\begin{itemize}
    \item ML-driven ABM analysis framework
    \item Systematic theory comparison approach
    \item Robust validation methodology
    \item Reproducible research pipeline
\end{itemize}

\end{document} 