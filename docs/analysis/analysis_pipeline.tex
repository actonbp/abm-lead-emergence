% Analysis Pipeline Documentation
\documentclass[11pt]{article}
\usepackage[utf8]{inputenc}
\usepackage{amsmath}
\usepackage{graphicx}
\usepackage{hyperref}
\usepackage{listings}
\usepackage{xcolor}
\usepackage{algorithm}
\usepackage{algpseudocode}
\usepackage{tikz}
\usepackage{float}

\title{Leadership Emergence Analysis Pipeline}
\author{ABM Research Team}
\date{\today}

\begin{document}
\maketitle

\begin{abstract}
This document provides a comprehensive description of the analysis pipeline used in our leadership emergence agent-based modeling research. The pipeline combines agent-based simulations with machine learning techniques to explore parameter spaces, detect emergence patterns, and validate theoretical predictions. We detail the components, workflows, and methodological approaches used throughout the analysis process.
\end{abstract}

\section{Overview}
The analysis pipeline consists of three main components:
\begin{enumerate}
    \item Agent-Based Model Simulation
    \item Machine Learning Analysis
    \item Theory Validation
\end{enumerate}

\section{Agent-Based Model}
\subsection{Model Structure}
The base leadership emergence model represents a system of interacting agents with the following properties:
\begin{itemize}
    \item Leader Identity (LI): Agent's self-perception as a leader (0-100)
    \item Follower Identity (FI): Agent's self-perception as a follower (0-100)
    \item Leadership Characteristics: Inherent leadership traits
    \item Implicit Leadership Theory (ILT): Mental model of leadership
\end{itemize}

\subsection{Interaction Dynamics}
During each simulation step:
\begin{algorithm}[H]
\caption{Agent Interaction Process}
\begin{algorithmic}[1]
\State Select random agents $i$ and $j$
\If{$i$.decidesToClaim()}
    \If{$j$.evaluatesGrant($i$)}
        \State $i$.increaseLeaderIdentity()
        \State $j$.increaseFollowerIdentity()
    \Else
        \State $i$.decreaseLeaderIdentity()
    \EndIf
\EndIf
\end{algorithmic}
\end{algorithm}

\section{Machine Learning Pipeline}
\subsection{Parameter Space Exploration}
The pipeline employs Bayesian optimization to explore the parameter space efficiently:

\begin{itemize}
    \item Initial sampling using Latin Hypercube method
    \item Gaussian Process surrogate model
    \item Expected Improvement acquisition function
\end{itemize}

Key parameters explored:
\begin{itemize}
    \item Number of agents: [4, 16]
    \item Leadership identity change rate: [0.1, 5.0]
    \item Claim threshold: [0.3, 0.7]
    \item Grant threshold: [0.4, 0.8]
    \item Schema weight: [0.0, 1.0]
    \item Social identity influence: [0.0, 1.0]
\end{itemize}

\subsection{Pattern Detection}
The pattern detection process involves:

\begin{enumerate}
    \item Feature Extraction
    \begin{itemize}
        \item Mean leader/follower identities
        \item Identity variances
        \item Emergence speed metrics
        \item Stability measures
    \end{itemize}
    
    \item Dimensionality Reduction
    \begin{itemize}
        \item Principal Component Analysis (PCA)
        \item Retention of components explaining 95\% variance
    \end{itemize}
    
    \item Clustering
    \begin{itemize}
        \item K-means clustering (k=3)
        \item Cluster analysis and characterization
    \end{itemize}
\end{enumerate}

\section{Theory Validation}
\subsection{Theoretical Frameworks}
We validate simulation results against three theoretical perspectives:

\begin{enumerate}
    \item Social Interactionist Perspective (SIP)
    \begin{itemize}
        \item Moderate emergence speed (0.6)
        \item High stability (0.8)
        \item Clear hierarchy (0.7)
    \end{itemize}
    
    \item Social Cognitive Perspective (SCP)
    \begin{itemize}
        \item Fast emergence speed (0.8)
        \item Moderate stability (0.6)
        \item Very clear hierarchy (0.8)
    \end{itemize}
    
    \item Social Identity Theory (SIT)
    \begin{itemize}
        \item Slower emergence speed (0.4)
        \item Very high stability (0.9)
        \item Moderate hierarchy clarity (0.6)
    \end{itemize}
\end{enumerate}

\subsection{Validation Metrics}
Key metrics for theory validation:
\begin{itemize}
    \item Emergence Speed: Time to stable structure
    \item Stability: Variance in leader identities over time
    \item Hierarchy Clarity: Separation between leader and follower roles
    \item Role Differentiation: Independence of leader/follower identities
\end{itemize}

\section{Analysis Workflow}
\subsection{Setup Phase}
\begin{enumerate}
    \item Configuration loading
    \item Pipeline initialization
    \item Parameter space definition
\end{enumerate}

\subsection{Exploration Phase}
\begin{enumerate}
    \item Initial parameter sampling
    \item Base model simulations
    \item Feature extraction and pattern analysis
\end{enumerate}

\subsection{Iterative Analysis}
For each iteration:
\begin{enumerate}
    \item Pattern detection
    \item Theory validation
    \item Surrogate model update
    \item Next parameter selection
    \item Additional simulations
\end{enumerate}

\subsection{Final Analysis}
\begin{enumerate}
    \item Comprehensive pattern analysis
    \item Theory alignment assessment
    \item Parameter importance ranking
    \item Visualization generation
\end{enumerate}

\section{Output Analysis}
\subsection{Pattern Analysis Results}
The pipeline produces detailed pattern analysis including:
\begin{itemize}
    \item Cluster characteristics
    \item Pattern frequencies
    \item Transition dynamics
\end{itemize}

\subsection{Theory Alignment}
For each theoretical framework:
\begin{itemize}
    \item Overall alignment score
    \item Metric-specific scores
    \item Comparative analysis
\end{itemize}

\subsection{Parameter Importance}
Analysis of parameter influence:
\begin{itemize}
    \item Relative importance rankings
    \item Interaction effects
    \item Sensitivity analysis
\end{itemize}

\section{Conclusion}
This analysis pipeline provides a systematic approach to:
\begin{itemize}
    \item Explore leadership emergence conditions
    \item Validate theoretical predictions
    \item Identify key parameters
    \item Discover novel emergence patterns
\end{itemize}

The combination of agent-based modeling, machine learning, and theory validation creates a robust framework for understanding leadership emergence dynamics.

\end{document} 